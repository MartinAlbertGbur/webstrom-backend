\titulka{zimného}{42.}

\begin{center}
\vskip -1.00cm
\textit{\textbf{Nezabudni si vytvoriť či aktualizovať profil na}  \url{https://seminar.strom.sk}.}
\end{center}

%\medskip
\znak{1}{Prvá}{23.~10.~2017}
\vskip -0.75cm


\uloha{1.}{Nech $a$, $b$, $c$ sú dĺžky strán trojuholníka $ABC$ ($a$ oproti $A$, $b$ oproti $B$ a $c$ oproti $C$). Navyše tieto dĺžky sú celočíselné a $b$, $c$ sú nesúdeliteľné čísla. Nech $D$ je priesečník strany $BC$ a osi uhla $BAC$. Dokážte, že ak sú trojuholníky $DBA$ a $ABC$ podobné, tak $c$ je druhou mocninou celého čísla.}

\uloha{2.}{Nech $P_1(x)=x^2+a_1x+b_1$ a $P_2(x)=x^2+a_2x+b_2$ sú dva kvadratické polynómy s celočíselnými koeficientami. Platí, že $a_1\not =a_2$ a existujú rôzne celé čísla $m$, $n$ také, že $P_1(m)=P_2(n)$ a $P_1(n)=P_2(m)$. Dokážte, že $a_1-a_2$ je párne.}

\uloha{3.}{Na stretnutie prišlo $2k+1$ ľudí. Každí dvaja sa buď poznajú, alebo nepoznajú (vzťahy sú vzájomné). Pre každú skupinu práve $k$ ľudí existuje človek mimo tejto skupiny, ktorý v nej pozná každého. Dokážte, že na stretnutí je človek, ktorý pozná všetkých.}

\uloha{4.}{Nech $c$, $d$ sú dva delitele čísla $n$, pričom $c>d$. Dokážte, že $c>d+d^2/n$.}

\uloha{5.}{Daný je trojuholník $ABC$. Nech $k$ je jeho pripísaná kružnica, ktorá sa dotýka strany $BC$ v bode $K$ a polpriamok $AB$ a $AC$ sa dotýka v bodoch $L$ a $M$. Kružnica s priemerom $BC$ pretína úsečku $LM$ v bodoch $P$, $Q$, pričom $P$ leží medzi $L$ a $Q$. Dokážte, že polpriamky $BP$ a $CQ$ sa pretínajú v strede kružnice $k$.}

\uloha{6.}{Je možné napísať do každého políčka nekonečnej štvorčekovej tabuľky prirodzené číslo tak, aby pre každú dvojicu prirodzených čísel $m$, $n$ platilo, že súčet čísel v ľubovoľnom obdĺžniku $m\times n$ je deliteľný $m+n$?
Poznámka: Tabuľka je nekonečná do všetkých štyroch smerov, môžeme si to predstaviť tak, že má riadky aj stĺpce očíslované celými číslami.}

\smallskip

%\vfill
%\pagebreak
\znak{2}{Druhá}{20.~11.~2017}
\vskip -0.75cm

\uloha{1.}{Celé čísla $a,\ b,\ c$ spĺňajú rovnosť $a+b+c=bc$. Dokážte, že číslo $(a+b)(a+c)$ je deliteľné 4.}

\uloha{2.}{Deti sú rozdelené do 3 tímov -- červeného, zeleného a modrého. Na začiatku je v červenom tíme $c$ detí, v zelenom $z$ detí a v modrom $m$ detí. Keď sa stretnú dve deti z rôznych tímov, tak sa obaja pridajú k tímu tej farby, ktorú nemal ani jeden z nich. V závislosti od $c$, $m$, $z$ zistite, či je možné, aby po čase skončili všetky deti v jednom tíme.}

\uloha{3.}{Biliardový stôl má tvar obdĺžnika $ABCD$. Nech $XKLMNX$ je dráha biliardovej gule, ktorá sa z bodu $X$ dostane po odraze od všetkých štyroch strán biliardového stola naspäť na pôvodné miesto, t.j. body $K,\ L,\ M,\ N$ ležia postupne na stranách $AB,\ BC,\ CD,\ DA$. Dokážte, že $KLMN$ je rovnobežník a dĺžka cesty nezávisí od polohy bodu $X$. Vyjadrite ju.}

\uloha{4.}{Na tabuli je napísaných $n>3$ rôznych prirodzených čísel, ktoré sú nanajvýš $(n-1)!$. Pre každú dvojicu čísel $a>b$ na tabuli si do zošita zapíšeme čiastočný podiel (výsledok po celočíselnom delení) čísel $a$ a $b$. (Teda ak $a=47$ a $b=7$, tak si zapíšeme 6.) Dokážte, že sme si do zošita zapísali aspoň 2 rovnaké čísla.}

\uloha{5.}{Nech $\alpha$ je dané reálne číslo. Nájdite všetky funkcie $f:\mathbb{R}\rightarrow \mathbb{R}$ také, že pre všetky reálne čísla $x$, $y$ platí
$$f(f(x+y)f(x-y))=x^2+\alpha yf(y).$$}

\vskip -0.65cm
\uloha{6.}{$ABCD$ je rovnobežník s ostrým uhlom $DAB$. Body $A,\ P,\ B,\ D$ ležia na jednej kružnici v tomto poradí. Priamky $AP$ a $CD$ sa pretínajú v bode $Q$. Bod $O$ je stred kružnice opísanej trojuholníku $CPQ$. Dokážte, že ak $D \neq O$, tak priamky $AD$ a $DO$ sú na seba kolmé.}

\vfill
\textbf{Autori zadaní úloh:} Florián Hatala, Matúš Hlaváčik, Henrieta Micheľová, Roman Staňo, Peter Kovács, Daniel Onduš, Martin Vodička, Jakub Genči