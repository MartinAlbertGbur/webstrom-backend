%%%%%%%%%%%%%%%%%%%%%%%%%%%%%%%%%%%%%%%%%%%%%%%%%%%%%%%%%%%%%%%%%%%%%%%%%%%%%%%%

\newcommand{\uloha}{

\ifnum\value{seria}=1

\ifcase\numexpr\value{uloha}-1

%%%%%%%%%%%%%%%%%%%%%%%%%%%%%%%%%%%%%%%%%%%%%%%%%%%%%%%%%%%%%%%%%%%%%%%%%%%%%%%%
% 1. séria
%%%%%%%%%%%%%%%%%%%%%%%%%%%%%%%%%%%%%%%%%%%%%%%%%%%%%%%%%%%%%%%%%%%%%%%%%%%%%%%%

{
Majme dotyčnicový štvoruholník $ABCD$ rozdelený uhlopriečkou $AC$ na 2 trojuholníky. Dokážte, že kružnica vpísaná trojuholníku $ABC$ sa dotkne úsečky $AC$ v rovnakom bode ako kružnica vpísaná trojuholníku $ADC$.
}

%%%%%%%%%%%%%%%%%%%%%%%%%%%%%%%%%%%%%%%%%%%%%%%%%%%%%%%%%%%%%%%%%%%%%%%%%%%%%%%%

\or

%%%%%%%%%%%%%%%%%%%%%%%%%%%%%%%%%%%%%%%%%%%%%%%%%%%%%%%%%%%%%%%%%%%%%%%%%%%%%%%%

{
Majme štvorcovú tabuľku s rozmermi $n \times n$. Každé políčko tabuľky je zafarbené červenou, zelenou alebo žltou farbou. Bez toho, aby sme videli tabuľku, si musíme tipnúť, či je v nej párny alebo nepárny počet červených políčok. Čo si máme tipnúť v závislosti od $n$, aby sme mali väčšiu šancu, že si tipneme správne? Vieme, že každé možné ofarbenie tabuľky je rovnako pravdepodobné.
}

%%%%%%%%%%%%%%%%%%%%%%%%%%%%%%%%%%%%%%%%%%%%%%%%%%%%%%%%%%%%%%%%%%%%%%%%%%%%%%%%

\or

%%%%%%%%%%%%%%%%%%%%%%%%%%%%%%%%%%%%%%%%%%%%%%%%%%%%%%%%%%%%%%%%%%%%%%%%%%%%%%%%

{
Konvexný 2020-uholník má všetky svoje vrcholy v mrežových bodoch (teda majú celočíselné súradnice) a má celočíselné strany. Dokážte, že obvod tohto útvaru je párne číslo.
}

%%%%%%%%%%%%%%%%%%%%%%%%%%%%%%%%%%%%%%%%%%%%%%%%%%%%%%%%%%%%%%%%%%%%%%%%%%%%%%%%

\or

%%%%%%%%%%%%%%%%%%%%%%%%%%%%%%%%%%%%%%%%%%%%%%%%%%%%%%%%%%%%%%%%%%%%%%%%%%%%%%%%

{
Medzi všetkými nezápornými číslami reprezentovanými vzťahom $36^k-5^l$, kde $k$ a $l$ sú kladné celé čísla, nájdite najmenšie. Svoje tvrdenie dokážte.
}
%%%%%%%%%%%%%%%%%%%%%%%%%%%%%%%%%%%%%%%%%%%%%%%%%%%%%%%%%%%%%%%%%%%%%%%%%%%%%%%%

\or

%%%%%%%%%%%%%%%%%%%%%%%%%%%%%%%%%%%%%%%%%%%%%%%%%%%%%%%%%%%%%%%%%%%%%%%%%%%%%%%%

{
V rovine je bod s celočíselnými súradnicami $[x,y]$, avšak tieto súradnice nepoznáme. Poznáme však hodnoty výrazov $x^2+y$ a $y^2+x$, pričom tieto hodnoty sú rôzne. Dokážte, že s týmito informáciami vieme jednoznačne určiť súradnice hľadaného bodu.
}
%%%%%%%%%%%%%%%%%%%%%%%%%%%%%%%%%%%%%%%%%%%%%%%%%%%%%%%%%%%%%%%%%%%%%%%%%%%%%%%%

\or

%%%%%%%%%%%%%%%%%%%%%%%%%%%%%%%%%%%%%%%%%%%%%%%%%%%%%%%%%%%%%%%%%%%%%%%%%%%%%%%%

{
Majme $k$ prepínačov v rade. Každý prepínač ukazuje hore, doprava, dole alebo doľava. Ak tri susedné prepínače ukazujú rôznymi smermi, prepneme všetky tri do štvrtého smeru. Ak by v jednom momente bolo viac takýchto trojíc, prepneme tú najviac naľavo. 
Ukážte, že sa proces zastaví.
}

%%%%%%%%%%%%%%%%%%%%%%%%%%%%%%%%%%%%%%%%%%%%%%%%%%%%%%%%%%%%%%%%%%%%%%%%%%%%%%%%

\fi

\else

\ifcase\numexpr\value{uloha}-1

%%%%%%%%%%%%%%%%%%%%%%%%%%%%%%%%%%%%%%%%%%%%%%%%%%%%%%%%%%%%%%%%%%%%%%%%%%%%%%%%
% 2. séria
%%%%%%%%%%%%%%%%%%%%%%%%%%%%%%%%%%%%%%%%%%%%%%%%%%%%%%%%%%%%%%%%%%%%%%%%%%%%%%%%

{
Majme čísla od 1 do $n$. Pre každé $n$ nájdite najväčšie $k$ také, že naše čísla vieme rozdeliť do $k$ skupín s rovnakým súčtom.
}
%%%%%%%%%%%%%%%%%%%%%%%%%%%%%%%%%%%%%%%%%%%%%%%%%%%%%%%%%%%%%%%%%%%%%%%%%%%%%%%%

\or

%%%%%%%%%%%%%%%%%%%%%%%%%%%%%%%%%%%%%%%%%%%%%%%%%%%%%%%%%%%%%%%%%%%%%%%%%%%%%%%%

{
Majme rovnostranný trojuholník. Každá jeho strana je rozdelená na $k$ rovnakých častí pomocou $k-1$ bodov. Týmito bodmi veďme rovnobežky so zvyšnými dvoma stranami trojuholníka. Takto vznikne trojuholníková sieť zložená z $k^2$ menších trojuholníkových políčok. Nazvime reťaz takú sekvenciu políčok, že každé políčko je v nej zahrnuté maximálne raz a po sebe nasledujúce políčka majú spoločnú stranu. Aká je najdlhšia možná reťaz?
}
%%%%%%%%%%%%%%%%%%%%%%%%%%%%%%%%%%%%%%%%%%%%%%%%%%%%%%%%%%%%%%%%%%%%%%%%%%%%%%%%

\or

%%%%%%%%%%%%%%%%%%%%%%%%%%%%%%%%%%%%%%%%%%%%%%%%%%%%%%%%%%%%%%%%%%%%%%%%%%%%%%%%

{
Každé z čísel $a_1, a_2, \dots , a_n$ je rovné $1$ alebo $-1$ a platí 
$$a_1a_2a_3a_4 + a_2a_3a_4a_5 + a_3a_4a_5a_6 + \dots + a_{n-1}a_na_1a_2 + a_na_1a_2a_3 = 0$$
Dokážte, že $n$ je deliteľné 4.
}

%%%%%%%%%%%%%%%%%%%%%%%%%%%%%%%%%%%%%%%%%%%%%%%%%%%%%%%%%%%%%%%%%%%%%%%%%%%%%%%%

\or

%%%%%%%%%%%%%%%%%%%%%%%%%%%%%%%%%%%%%%%%%%%%%%%%%%%%%%%%%%%%%%%%%%%%%%%%%%%%%%%%

{
Je daný štvorsten $ABCD$. Po úsečke $AB$ sa pohybuje bod $X$. Označme $P$ pätu výšky spustenej z bodu $D$ na priamku $CX$. Určte množinu bodov $P$, ktoré vyhovujú zadaniu.
}
%%%%%%%%%%%%%%%%%%%%%%%%%%%%%%%%%%%%%%%%%%%%%%%%%%%%%%%%%%%%%%%%%%%%%%%%%%%%%%%%

\or

%%%%%%%%%%%%%%%%%%%%%%%%%%%%%%%%%%%%%%%%%%%%%%%%%%%%%%%%%%%%%%%%%%%%%%%%%%%%%%%%

{
Nájdite najväčšie číslo $p$ také, že je možné na šachovnicu $2019\times 2019$ umiestniť $p$ pešiakov a $p+2019$ veží tak, aby sa žiadne dve veže neohrozovali. (Dve veže sa ohrozujú, ak sú v tom istom riadku alebo stĺpci a všetky políčka medzi nimi sú prázdne).
}

%%%%%%%%%%%%%%%%%%%%%%%%%%%%%%%%%%%%%%%%%%%%%%%%%%%%%%%%%%%%%%%%%%%%%%%%%%%%%%%%

\or

%%%%%%%%%%%%%%%%%%%%%%%%%%%%%%%%%%%%%%%%%%%%%%%%%%%%%%%%%%%%%%%%%%%%%%%%%%%%%%%%

{
Nech $ABC$ je ostrouhlý nerovnoramenný trojuholník, $M$ je stred strany $BC$ a $AD$ je os uhla pri vrchole $A$, pričom $D$ leží na strane $BC$. Kružnica opísaná trojuholníku $ADM$ pretína $AB$ v bode $E$ a $AC$ v bode $F$. Bod $I$ je stred $EF$ a $MI$ pretína priamky $AB$ v bode $X$ a $AC$ v bode $Y$. Dokážte, že $AXY$ je rovnoramenný.
}

%%%%%%%%%%%%%%%%%%%%%%%%%%%%%%%%%%%%%%%%%%%%%%%%%%%%%%%%%%%%%%%%%%%%%%%%%%%%%%%%

\fi

\fi

}

%%%%%%%%%%%%%%%%%%%%%%%%%%%%%%%%%%%%%%%%%%%%%%%%%%%%%%%%%%%%%%%%%%%%%%%%%%%%%%%%

