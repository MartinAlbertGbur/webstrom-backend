\titulka{zimného}{43.}

\begin{center}
\vskip -1.00cm
\textit{\textbf{Nezabudni si vytvoriť či aktualizovať profil na}  \url{https://seminar.strom.sk}.}
\end{center}

%\medskip
\znak{1}{Prvá}{29.~10.~2018}
\vskip -0.75cm


\uloha{1.}{Dokážte, že ak sú $a$, $b$, $c$, $d$ reálne čísla a $ac=2(b+d)$, tak má aspoň jedna z rovníc $x^2 + ax +b=0$ a $x^2+cx+d=0$ všetky korene reálne.}

\uloha{2.}{Majme trojuholník $ABC$, kde $AB$ je najdlhšia jeho strana. Zvoľme bod $D$ tak, aby sa nachádzal na opačnej polpriamke k $BA$, teda $B$ je medzi bodmi $A$ a $D$ a zároveň platí $|BC|=|BD|$. Dokážte, že trojuholník $ACD$ je tupouhlý.}

\uloha{3.}{Máme $n$ bodov v rovine. Môžeme urobiť to, že dva z nich vyberieme a následne obidva presunieme do stredu úsečky, ktorá ich spája. Body iba presúvame, žiaden z nich nezaniká. Zistite, pre ktoré $n$ je možné vždy presúvať dané body tak, aby všetky splynuli (boli v jednom bode).}

%DOPLNIT o polynomoch do Mohlo by sa hodit..
\uloha{4.}{Nájdite všetky polynómy $P(x)$ s reálnymi koeficientami, ktoré spĺňajú $(x^2 - 6x + 8)\cdot P(x) = (x^2 + 2x)\cdot P(x-4)$ pre všetky reálne čísla $x$.}

%ZADANIE skontrolovat - plus neni mi zo zadania jasne, ci vsetkych tych 2n bodov musi tvorit kruznicu alebo nemusia byt vsetky? ci "s vrcholmi vo všetkých vybratých bodoch?" alebo "s vrcholmi v niektorých z vybratých bodov?"?
\uloha{5.}{Majme $n^2$ bodov rozmiestnených do mriežky $n\times n$. Z nich náhodne vyberieme $2n$ bodov. Dva vybrané body sú spojené zelenou úsečkou, ak sú v rovnakom riadku, a červenou úsečkou, ak sú v rovnakom stĺpci. Dokážte, že existuje uzavretá lomená čiara s vrcholmi vo vybraných bodoch a so stranami tvorenými týmito úsečkami taká, že sa farby jej strán striedajú.
%Je daná tabuľka $n\times n$, v ktorej je vybraných $2n$ políčok. Dve vybrané políčka sú spojené zelenou hranou, ak sú v rovnakom riadku, a červenou hranou, ak sú v rovnakom stĺpci. Dokážte, že existuje kružnica, v ktorej sa striedajú farby hrán. pozn. Hrana a kružnica sú v tomto prípade grafové pojmy
}

\uloha{6.}{Prirodzené číslo $n$ nazveme chutné, ak pre ľubovoľné dve prirodzené čísla $a$, $b$ také, že $a+b=n$ platí, že aspoň jeden zo zlomkov $\frac ab,\frac ba$ má konečný desatinný rozvoj. Existuje nekonečne veľa chutných čísel? Svoje riešenie odôvodnite.}

\smallskip

%\vfill
%\pagebreak
\znak{2}{Druhá}{3.~12.~2018}
\vskip -0.75cm

\uloha{1.}{Nech $ABCD$ je štvoruholník, v ktorom existuje kružnica, ktorá prechádza stredmi všetkých strán tohto štvoruholníka. Dokážte, že $AC$ a $BD$ sú na seba kolmé.}

\uloha{2.}{Máme riadok, v ktorom je 1000 čísel. Pod tento riadok pridáme ďalší tak, že pod každým číslom $a$ je napísaná hodnota $f(a)$, kde $f(a)$ je počet výskytov čísla $a$ v predchádzajúcom riadku. Takto postupne pridávame ďalšie riadky. Dokážte, že po pridaní dostatočného počtu riadkov budú pod sebou dva rovnaké riadky.}

%netreba dodat co je faktorial? viem ze sa zvyklo pisat, ale neviem ci aj v strome
\uloha{3.}{Je daná rovnica $x!y!z!=t!$, kde $x$, $y$, $z$, $t$ sú prirodzené čísla väčšie ako jedna. Ukážte, že existuje nekonečne veľa štvoríc $(x,y,z,t)$, ktoré spĺňajú túto rovnicu. \it{Pozn.: $n!$ je číslo $n\cdot (n-1)\cdot \dots \cdot 2\cdot 1$.}}

\uloha{4.}{Majme $n$ nenulových čísel, ktorých súčet je 0. Ukážte, že je možné ich očíslovať tak, aby platila nerovnosť: $$a_1a_2 + a_2a_3 + \dots + a_na_1 < 0.$$}

\vskip -0.65cm
\uloha{5.}{Dokážte, že ľubovoľné celé číslo môže byť napísané ako súčet 5 tretích mocnín celých čísel.}

\uloha{6.}{Nech $ABC$ je ostrouhlý trojuholník s $|AB|<|AC|$. Nech $M$, $N$ sú postupne stredy strán $AB$ a $AC$ a nech $AD$ je výška v tomto trojuholníku. Na úsečke $MN$ zvolíme taký bod $K$, že $|BK|=|CK|$. Polpriamka $KD$ pretína kružnicu opísanú trojuholníku $ABC$ v bode $Q$. Dokážte, že body $C$, $N$, $K$, $Q$ ležia na jednej kružnici.}

\vfill
\textbf{Autori zadaní úloh:} Žaneta Semanišinová, Martin Masrna, Kristína Mišlanová, Roman Staňo, Peter Kovács, Daniel Onduš, Jakub Genči