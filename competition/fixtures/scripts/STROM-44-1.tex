\titulka{zimného}{44.}

\begin{center}
\vskip -1.00cm
\textit{\textbf{Nezabudni si vytvoriť či aktualizovať profil na}  \url{https://seminar.strom.sk}.}
\end{center}

%\medskip
\znak{1}{Prvá}{28.~10.~2019}

\vskip -0.75cm



%predejit na nieco ine ako mravca, aj ked ta vec co sa da vygooglit neni vobec rovnaka uloha
%som predejil akurat neviem ci to dava zmysel
%som to trochu upravila, pozrite ci lepsie (Zanet)
%ok
\uloha{1.}{
Baník sa nachádza na kocke. Každú minútu sa rozhodne, či prejde do niektorého vrcholu, ktorý susedí hranou s vrcholom, kde sa práve nachádza, alebo sa prevŕta do vrcholu, ktorý je presne oproti. Baník sa vždy rozhodne náhodne s rovnakou pravdepodobnosťou pre každú voľbu nasledujúceho vrcholu. Aká je pravdepodobnosť, že sa po 2019 minútach bude nachádzať vo vrchole, ktorý je presne oproti počiatočnému vrcholu?
}

% je to takto myslene? dodala som tam "vo vyrazoch plus minus" (Zanet)
%mozno este specifikovat n
%a este som pridal aj xko na koniec
\uloha{2.}{Dokážte, že pre všetky $x \in \mathbb{R}$, pre všetky celé kladné $n$ a pre ľubovoľné rozdelenie znamienok $+$ a $-$ vo výrazoch $\pm$ platí
$$x^{2n} \pm x^{2n-1} + x^{2n-2} \pm x^{2n-3} + x^{2n-4} \pm \dots \pm x + 1 > \frac{1}{2}.$$
}


%vygooglil som ale len ked som vedel zadanie po anglicky, asi je to safe
% krasne zadejene, dava zmysel (Zanet)
\uloha{3.}{
Majme deku s rozmermi $3 \times 3$ metre ofarbenú troma farbami. Ukážte, že vieme zapichnúť dvojzubec do deky tak, že hroty dvojzubca prepichnú deku na miestach s rovnakou farbou. Predpokladáme, že hrot je jeden bod a že vzdialenosť hrotov dvojzubca je $1 cm$.
}

%vygooglil som a neviem co s tym vieme robit
%pozmenene zadanie
% ok (Zanet)
\uloha{4.}{
Dokážte, že ak $a$, $b$ sú korene polynómu $x^2 - 8x + 1$, tak potom pre všetky nezáporné celé čísla $n$ platí, že $a^n +b^n$ je celé číslo nedeliteľné siedmimi.
}


%za mna uz ok (Dano)
\uloha{5.}{
Daná je úsečka $PQ$ a kružnica $k$. Body $A$ a $B$ sa hýbu po kružnici $k$ tak, aby vždy platilo $|AB|=|PQ|$. Ak označíme $T$ priesečník osí úsečiek  $AP$ a $BQ$, potom každá poloha tetivy $AB$ určuje polohu bodu $T$. Dokážte, že všetky možné body $T$ ležia na jednej priamke. 
}


%zakomentovane je stare zadanie, mozete si precitat nove
\uloha{6.}{
Každý bod v rovine s celočíselnými súradnicami je ofarbený buď červenou alebo modrou farbou tak, aby boli splnené podmienky:
\begin{enumerate}
\item Na úsečke spájajúcej červené body neleží žiaden modrý bod.
\item Ak majú dva modré body vzdialenosť 2, potom bod uprostred medzi nimi je modrý.
\end{enumerate}
%Dokážte, že trojuholník s červenými vrcholmi neobsahuje žiadne modré body.
Dokážte, že z ľubovoľného červeného bodu sa vieme dostať do ľubovoľného iného tak, že nemusíme prejsť cez žiaden modrý bod, pričom kroky vieme robiť len vodorovne a zvislo, vždy o vzdialenosť jedna.
}

\newpage

%%%%%%%%%%%%%%%%%%%%%%%%%%%%%%%%%%%%%%%%%%%%%%%%%% ZS - Druha seria %%%%%%%%%%%%%%%%%%%%%%%%%%%%%%
\znak{2}{Druhá}{2.~12.~2019}
\vskip -0.75cm


%toto bude asi tazsie vygooglit jak vyriesit
% to je myslené tak, že existuju i a j? a dala by som pocet vsetkych cisel j ak je to tak myslene (Zanet)
%uz je to asi ok (Dano)

\uloha{1.}{
Určte, pre ktoré kladné celé čísla $n$ existuje tabuľka $n\times n$ obsahujúca $n^2$ kladných celých čísel, pre ktorú platí, že pre ľubovoľnú voľbu $i$ a $j$ (môžu nadobúdať hodnoty od $1$ po $n$) je v políčku v $i$-tom riadku a $j$-tom stĺpci počet všetkých hodnôt $j$, ktoré sa vyskytujú v $i$-tom stĺpci. 
}

%treba predejit uzasne cislo (v originali je to vtipne btw), a navyse som vygooglil, ale to pojde predejit ked sa zmeni nko na x napriklad a prepise sa veta
%som predejil, snad je to safe, akurat som to musel zmenit az na m, ta snad je to ok ze m ako mihal 
%za mna ok (Zanet)
\uloha{2.}{
Mihál nemá rád čísla s prívlastkom. Má však rád také kladné celé čísla $m$, pre ktoré je každé z čísel $m$, $m+1$, $m+2$ a $m+3$ deliteľné svojim ciferným súčtom. Dokážte, že ak posledná cifra v takomto čísle je 8, tak potom predposledná cifra tohto čísla je nutne 9.
}

%ok (Zanet)
\uloha{3.}{
Majme štvorec $ABCD$, ktorý má nad stranou $AB$ zostrojenú polkružnicu vo vnútri štvorca $ABCD$. K tejto polkružnici veďme dotyčnicu prechádzajúcu bodom $C$ rôznu od priamky $CB$ a označme jej bod dotyku $F$. Prienik úsečky $BD$ a polkružnice označíme $E$. Aký je obsah trojuholníka $BEF$, ak je dĺžka strany štvorca $ABCD$ rovná 10?
}

%dopisat (neni uz nahodou?) 
% Podla mna je
%co znamena stena veze? vandali sprejuju priamku na tom valci? (Zanet) 
%ono je to 2d uloha pri pohlade zhora takze ano
%s tou poslednou verziou sme uz celkom spokojni (Dano)
\uloha{4.}{
%V odľahlej časti mesta máme niekoľko rovnakých veží s kruhovým pôdorysom. Vandali posprejujú stenu veže práve vtedy, keď z daného miesta nevidno žiadnu inú vežu. Dokážte, že celková dĺžka posprejovaných častí je rovná obvodu jednej veže. 
%V odľahlej časti mesta máme niekoľko rovnakých veží s kruhovým pôdorysom. Vandali nasprejujú na stenu veže zvislú priamku práve vtedy, keď z daného miesta nevidno žiadnu inú vežu. Dokážte, že celková dĺžka posprejovaných častí po obvode veží je rovná obvodu jednej veže. 
V odľahlej časti mesta stojí niekoľko rovnakých veží s kruhovým pôdorysom. Vandali sa rozhodujú, kde budú sprejovať, pričom na mape si vyznačia bod na obvode veže práve vtedy, keď z daného miesta nevidno žiadnu inú vežu. Dokážte, že celková dĺžka vyznačených oblastí je rovná obvodu jednej veže. 
}

%asi ok (Zanet)
\uloha{5.}{
Dvaja hráči hrajú piškvorky na nekonečne veľkom trojuholníkovom papieri a striedajú sa v ťahoch. Ten, kto je na ťahu, vždy nakreslí svoju značku do niektorého voľného políčka. Vyhrá hráč, ktorý má ako prvý neprerušovanú rovnú radu (smerujúcu jedným z troch možných smerov v mriežke) aspoň $n$ svojich znakov, kde $n$ je nejaké prirodzené číslo. V závislosti na $n$ určte, kto má vyhrávajúcu alebo neprehrávajúcu stratégiu.
}

%asi ok (Zanet)
\uloha{6.}{
Nájdite všetky funkcie $f: \mathbb R \rightarrow \mathbb R$ také, že pre všetky reálne čísla $x, y$ platí: $f(xy+f(x))=xf(y)$.
}


\vfill

\textbf{Autori zadaní úloh:} Žaneta Semanišinová, Martin Masrna, Kristína Mišlanová, Roman Staňo, Peter Kovács, Daniel Onduš, Jakub Genči
